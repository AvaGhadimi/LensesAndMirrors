\documentclass[11pt, a4paper]{article} 

\usepackage[english]{babel} % English language hyphenation
%\usepackage{microtype} % Better typography
\usepackage{amsmath,amsfonts,amsthm} % Math packages for equations
\usepackage{graphicx} % Required for adding images
\usepackage{geometry}
\usepackage{siunitx}
\usepackage{todonotes}
\geometry{%
	top=1cm, % Top margin
	bottom=1.5cm, % Bottom margin
    left=3cm, % Left margin
    right=3cm, % Right margin
	includehead, % Include space for a header
	includefoot, % Include space for a footer
    %showframe, % Uncomment to show how the type block is set on the page
}
\usepackage{colortbl,booktabs}

% no paragraph indent, but small skip
\setlength{\parindent}{0pt}
\setlength{\parskip}{\baselineskip}

% ----------------------------------------
%            Title etc
% ----------------------------------------
%newtoks usage:
%\institution{Bama}
%\the\institution
\newtoks\institution


% ----------------------------------------
%            HEADER/FOOTER
% ----------------------------------------
\usepackage{fancyhdr}
\pagestyle{fancy}
\fancyhf{}
\makeatletter 
\lhead{\@title}
\makeatother 
%\rhead{}
\rfoot{Page \thepage}
\lfoot{\the\institution}
\renewcommand{\headrulewidth}{0pt} % no headrule (line)
\renewcommand{\footrulewidth}{.5pt} % 

% ----------------------------------------
%            ANSWER LINES
% code from the exam.cls class
% see https://tex.stackexchange.com/questions/117548/exam-lines-not-with-the-exam-package
% ----------------------------------------

\makeatletter
\newlength\linefillheight
\newlength\linefillthickness
\setlength\linefillheight{.25in}
\setlength\linefillthickness{0.1pt}

\newcommand\linefill{\leavevmode
    \leaders\hrule height \linefillthickness \hfill\kern\z@}


\def\fillwithlines#1{%
  \begingroup
  \ifhmode
    \par
  \fi
  \hrule height \z@
  \nobreak
  \setbox0=\hbox to \hsize{\hskip \@totalleftmargin
          \vrule height \linefillheight depth \z@ width \z@
          \linefill}%
  % We use \cleaders (rather than \leaders) so that a given
  % vertical space will always produce the same number of lines
  % no matter where on the page it happens to start:
  \cleaders \copy0 \vskip #1 \hbox{}%
  \endgroup
}
\makeatother


% ----------------------------------------
%       Negative phantom
% http://www.math.mcgill.ca/rags/texmacros/nphantom.tex
% ----------------------------------------

\usepackage{array,tabularx}

\catcode`@=11

\def\nvphantom{\v@true\h@false\nph@nt}
\def\nhphantom{\v@false\h@true\nph@nt}
\def\nphantom{\v@true\h@true\nph@nt}
\def\nph@nt{\ifmmode\def\next{\mathpalette\nmathph@nt}%
  \else\let\next\nmakeph@nt\fi\next}
\def\nmakeph@nt#1{\setbox\z@\hbox{#1}\nfinph@nt}
\def\nmathph@nt#1#2{\setbox\z@\hbox{$\m@th#1{#2}$}\nfinph@nt}
\def\nfinph@nt{\setbox\tw@\null
  \ifv@ \ht\tw@\ht\z@ \dp\tw@\dp\z@\fi
  \ifh@ \wd\tw@-\wd\z@\fi \box\tw@}

 % Specifies the document structure and loads requires packages
\title{Lenses and Mirrors}
\date{}
\author{}
\institution{Department of Physics and Astronomy, University of Alabama}

\begin{document}

% ---- Title stuff -----------------------
\thispagestyle{empty}
\noindent
\begin{tabularx}{\textwidth}{ll>{\raggedleft\arraybackslash}Xr}
    Course and Section: & \underline{\hspace{2cm}} & Names: & \underline{\hspace{6cm}}\\[0.3cm]
    Date: & \underline{\hspace{2cm}} & & \underline{\hspace{6cm}}\\[0.3cm]
    & & TA name: & \underline{\hspace{6cm}}
\end{tabularx}
\bigskip
\begin{center}
    {\Huge \bf Lenses and Mirrors}\\
\end{center}
% ----------------------------------------
\begin{center}
\textsc{Answer all questions from the text on the lines beneath.}
\end{center}
In this experiment you will study lenses and mirrors by creating images on screens.
By doing so you will study the laws describing them.\todo{rewrite}

\section{Equipment}
\begin{minipage}{0.49\textwidth}
    \begin{itemize}
        \item Optical bench
        \item Converging lens, diverging lens
        \item Light source (as an object)
    \end{itemize}
\end{minipage}
\begin{minipage}{0.49\textwidth}
    \begin{itemize}
        \item Viewing screen
        \item Mirror
        \item Half-screen
    \end{itemize}
\end{minipage}

\section{Procedure}
\subsection{Converging lens}
Take a \textit{converging lens} and examine it.
Is it thicker or thinner in the center?
What do objects look like through it?
Could it be used as a magnifying glass?
Try the lens both close to objects and far away.
\fillwithlines{3cm}
If a lens is used to form an image of something infinitely far away, the
distance from the lens to the image is defined to be the focal length. 
This is because the rays from an infinitely remote object are essentially
parallel, thus converging at the focal point.

Use the lens to form an image of “something far away”. 
To do this, place the lens on the optical bench and move the screen on the
bench until a sharp image is formed on it. 
\missingfigure{Converging lens picture, parallel light rays falling in, meeting at focal point}
Then read off the focal length using the scale on the optical bench.
\begin{equation*}
    \text{Focal length} = \underline{\hspace{2cm}}\si{cm}
\end{equation*}

\subsection{Image-Object Relationship of a Converging Lens}
In this part, use the light box as the object; the lens will form an image of
this box on the screen.
Place the object and the screen at opposite ends of the bench such that the
distance between them is 110 cm.
Place a white sheet of paper in front of the white screen. 
Move the lens between them until a sharp image is formed on the screen.
You should note that there are two positions for the lens which give sharp
images. \textit{Why} is that so?
\fillwithlines{3cm}

Take measurements for each of these positions.
Repeat these measurements for $L = 100~\si{cm}$ and $L = 90~\si{cm}$.
Record the distance $p$ from the lens to the object and the distance
$q$ from the lens to the image.
Also measure the height $h'$ of the image and the height $h$
of the object.
\begin{equation*}
    h = \underline{\hspace{2cm}}\si{cm}
\end{equation*}

\begin{center}
    \def\arraystretch{1.5}
    \setlength\tabcolsep{0.6cm}
    \begin{tabular}{|c|c|c|c|c|c|c|}
        \hline
        \rowcolor{gray!40}
        $L$ (cm) & $p$ (cm) & $q$ (cm) & $h'$ (cm) & $f$ (cm) & $h'/h$ & $q/p$ \\
        \hline
        110 & ~ & ~ & ~ & ~ & ~ & ~ \\
        \hline
        110 & ~ & ~ & ~ & ~ & ~ & ~ \\
        \hline
        100 & ~ & ~ & ~ & ~ & ~ & ~ \\
        \hline
        100 & ~ & ~ & ~ & ~ & ~ & ~ \\
        \hline
        90 & ~ & ~ & ~ & ~ & ~ & ~ \\
        \hline
        90 & ~ & ~ & ~ & ~ & ~ & ~ \\
        \hline
    \end{tabular}
\end{center}

\missingfigure{Show what p,q,f are}
The equation relating the image and object distances of a \textit{thin lens} is
\begin{equation}
    \frac{1}{p} + \frac{1}{q} = \frac{1}{f}\,.
    \label{eq:f}
\end{equation}
Use equation~(\ref{eq:f}) to calculate the \textit{focal length} $f$ from each set of data in the table
and fill in that part of the table.
\begin{itemize}
    \item Do you get consistent values for the \textit{focal length} $f$?
    \item Do these results agree with the focal length you found in the \textit{first part} of the experiment?
    \item Which method do you think is the most \textit{accurate}?
\end{itemize}
\fillwithlines{3cm}
Finally, look at the \textit{last two columns} of the table.
Do you find that the two ratios are equal?
If so, why?
\fillwithlines{3cm}
\subsection{Diverging Lens}
A diverging lens will produce a virtual image of a real object. 
To observe this, place the lens on the optical bench about 20 cm from the light
box.
Move the screen on the bench, can you find the image on the screen?
Now, look at the light box through the lens, can you see the image?
\begin{itemize}
    \item Is this image smaller or bigger than the object?
    \item Is the image you are viewing \textit{inverted}?
    \item Is it \textit{virtual}?
    \item Is it located in \textit{front} or in the \textit{back} of the lens?
\end{itemize}
\fillwithlines{3cm}


\missingfigure{measurement of f', diverging lens}
In order to produce a real image you also need to use converging lens.
This image can be produced by placing a converging lens between the diverging
lens and the source.
By suitably adjusting the distances of the two lenses and the screen, a real
image is produced on the screen.
Use this method to determine $f'$, the focal length of the diverging lens, as follows:

Using the thin lens equation, calculate distance of the image q formed by the converging lens
\begin{equation*}
    q = \underline{\hspace{2cm}} \si{cm}\,.
\end{equation*}
This image created by the converging lens, becomes the object for the diverging lens. 
Next, measure the distance $d$ between the two lenses.
\begin{equation*}
    d = \underline{\hspace{2cm}} \si{cm}\,.
\end{equation*}
The ``new'' object distance $p'$, with respect to the diverging lens, is given by the
difference between the two distances $q$ and $d$,
\begin{equation*}
    p' = q - d  = \underline{\hspace{2cm}} \si{cm}\,.
\end{equation*}
Measure the distance $q'$ between the screen and the diverging lens
\begin{equation*}
     q'  = \underline{\hspace{2cm}} \si{cm}\,.
\end{equation*}
Finally, using the thin lens equation for $p'$, $q'$ and $f'$,
\begin{equation}
    \frac{1}{p'} + \frac{1}{q'} = \frac{1}{f'}\,,
\end{equation}
find the focal length $f'$ of the diverging lens,
\begin{equation*}
     f'  = \underline{\hspace{2cm}} \si{cm}\,.
     \label{eq:fprime}
\end{equation*}
\textit{Note:} In equation~(\ref{eq:fprime}) $p'$ has to be negative, since the image formed by
the converging lens is behind the diverging lens.\todo{clarify}

\end{document}
